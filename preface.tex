\phantomsection
\chapter*{前\qquad 言}
\addcontentsline{toc}{chapter}{前言}
%%%%%%%%%%%%%%%%%%%%%%%%%%%%%%%%%%%%%%%%%%%%%%%%%%%%%%%%%%%%

《论语》是传统文化的宝贵财富,一个时代应该有一个时代的《论语》读本。

什么是传统文化?我想,它应该是让人发自内心感到熟悉亲切,感到温暖安乐的东西吧。一个国家的传统文化,是国民的精神港湾,闪耀着自古以来的共同理想。柳诒徵先生的《中国文化史》说:\lyq{孔子者,中国文化之中心也。无孔子则无中国文化。}如果不先了解孔子,又怎能断定这句话的是与非呢?
% NOTE: 柳书随后一句是:\lyq{自孔子以前数千年之文化,赖孔子以传;自孔子以后数千年之文化,赖孔子而开。}

《论语》是写孔子最真实可靠的材料。孔子在古代被称为\lylink{sheng4}{圣人},他自己却觉得,连\lylink{junzi}{君子}的标准都还\lylink{7.33}{够不上},兴致勃勃地做着\lylink{7.17}{学习计划}呢。孔子的形象,有时被神化,有时又被虚化,容易让人敬而远之,这可不是他的错。稍微读读《论语》就会觉得,孔子没有那么高深莫测,是一位可敬又风趣的老人家。仔细想想还会明白,他的心灵离我们有多么近,尽管年代隔了那么远。

《论语》是一本很小的书。我采用的\lylink{lunyuversion}{版本},原文分为20篇,共512章,15916字。如果按现代文的阅读速度,40分钟就能读完一遍。它的文字简练优美,只是有些难字需要解释,为了帮助理解,还要补充一些背景材料。它的涵义丰富深远,不太适合速读,应该反复玩味。这就是同时提供了注释版和\lylink{baiwenban}{白文版}的原因。
% NOTE: 字数统计,见 rawblobs.py。

《论语》的注本已经数不胜数,我又不是文科专业,为什么要另搞一套呢?

我非常喜欢《论语》。我常想,假如《论语》能说话,一定很羡慕“人家”《圣经》有那么多的优质资源,从网站到电子书,内容精良,还可以免费使用。语录体的《论语》,难道不是碎片时间看上几眼,反复揣摩的绝好选择?

我见过的注本中,古代的作品有的简洁有的细密,但现代人读起来难免艰涩。现代的作品,有的只解释字词而缺乏感想,比较沉闷,有的回护古人较多,略显执拗,有的刻意回避了传统的观点,缚手缚脚,也有的发挥较远,并不耐读。而且一旦出版,大小瑕疵无法随时修正,新材料新成果难以及时补充,让认真的作者和读者都很苦恼。
% NOTE: 当然,还有很多出色的注本,因为这样那样的原因,消失于大众的视野了,这也并不公平。

这本书是我在业余时间的一个尝试,希望写成一个既严谨又可读,既保留传统内涵又反映现代价值的普及读本。我没受过文史科班训练,遇到不懂的就去查考比较,力求字字落实。因为没有规范写法的条条框框,内容和形式的选择就相对灵活,这在\lylink{usage}{使用说明}里作了交代。把它发布为\lylink{projectinfo}{开源项目},就可以自由地阅读,便捷地更新,让更多的人直接受益。有时候人总要撒点野,做点自己特别想做的事情,这个项目就是一个例子。

孔子已经过时了吗?这是每一章都无法回避的问题,撰写和维护这本书也可以看成反复追问与应答的一种记录。我的想法是,既不刻意回护,也不苛求古人,如果言之成理,就尽量从正面解读每一章的意义。唱反调固然痛快,角度又多,但放在注本里,就显得单薄乏味。读者自己确实应该从多个角度揣摩,作为必不可少的思维练习。

当然,我再钟爱《论语》,也不认为2000多年前的古书是包治百病的神药。如果说传统文化\lylink{19.22}{未坠于地},就应该理解其内涵,发扬其精神,保持健康的传承。倘若孔子有知,应该会对《动物庄园》(\emph{Animal Farm})里老少校(Old Major)的话深有同感:\lyq{我岁数大了,嗓音又哑,可一旦教会你们这曲调,你们就能自己唱得更好。}(\lyqe{I am old and my voice is hoarse, but when I have taught you the tune, you can sing it better for yourselves.})

这本书尚处于初级阶段。我读书太少,眼界太窄,能投入的时间又有限,各种谬误也会相当可观。不过程序员的好处是,你指出来错误,他就一定会改,欢迎随时\lylink{projectinfo}{批评指正}。
% TODO: remove "这本书尚处于初级阶段。" after 3 passes of self-reviews, adding proper images, and (most importantly) finishing the guide.

衷心感谢教我语文的两位王老师和欧阳老师,教我英语的拾老师和田老师。如果不是你们的言传身教,很难想象我会一直保持阅读与思考的习惯。希望这本小书没有让你们失望。

\lycenterpar{阅读愉快!}

% NOTE: 前言初稿写于2016-3-9,人机围棋大战5番棋之首战,李世石九段186手负Google AlphaGo。目标:在电脑更懂欣赏之前,把这本书写完……
